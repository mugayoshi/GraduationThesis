\section{はじめに}
\label{sec:intro}
今日では,スマートフォンが多くのユーザによって使われており,その中でも Android 端末は世界中で最も普及している.
IDC が世界中のスマートフォンのシェアを調査したところ,Android 端末は 80 \%以上のシェアを占めていることがわかった\cite[osshare].
つまり,Android 端末は他の OS の端末,iOS, Windows Phone, Black Berry OS に比べて,より多くのユーザによって使われていることがわかる.
このように Android が普及しているのは,Android がオープンソースであり,だれでも Android のソースコードを手に入れられることが 1 つの要因である.
そのため,様々なメーカーによって開発が行われ,価格,スペックに関して多種多様な製品が世界中で販売されている.
%また,Android SDK が提供されているために,気軽に Android アプリケーションを Java で実装することができる.

しかし,Androidの普及に伴い,Android 端末を標的にしたAndroid アプリケーションのマルウェアによる被害が増えている.
先に述べたように,Android はオープンソースであるため,攻撃者は脆弱性を見つけることは 他のモバイル端末 OS に比べると容易である.
Cisco の 2014 年のレポート\cite{cisco} によると,モバイル向けマルウェアの 99.9\% が Android を標的にしていると報告している.
Y.Zhou, X.Jiang の研究\cite{dissect} では,彼らの研究に用いたデータセットの数は,2011 年の6月では 209 個だったのが,同年 10 月には 1260 個にも増加している.
2014 年の 9 月には,ロシアで銀行口座を狙った Android マルウェアを作成したとして,2名の逮捕者が出ている.
これらの事例を見てわかるように,Android 端末を標的にしたマルウェアによる被害は深刻であり,Android 端末のユーザは危険にさられている.

これまで解析された多くの Android マルウェアが次の 2 つの挙動を示している.
1 つは,個人,または端末の情報を抜き取ることである.
感染した端末の情報を内部で収集し,この情報を外部へ送信する.
例えば感染した端末に保存されている電話帳データを盗むことによって,マルウェア作成者は他の端末への攻撃が可能になる.
もうひとつは,ユーザへの不正な課金である.
バックグラウンドで Short Message Service (SMS) を送ることで,ユーザは不正に金額を請求させられてしまう.
さらにマルウェアはバックグラウンドでメッセージを送信しているためにユーザは請求が来るまで全く気づかない.
もちろん Android マルウェアはこれ以外にもさまざまな挙動を示す.
このような被害を少なくするためには Android を標的にしたマルウェアを解析することは重要である.

code.google.com が提供している既存のAndroid マルウェアの解析ツールとして androguard\cite{aguard}と droidbox\cite{dbox} の2つがある.
%Android アプリケーションは Java で実装されている.
androguard は Java で実装されている Android アプリケーションのコード解析を行うことで,アプリケーション内のクラスごとの関係を示すグラフを作る.
そして,無害なノードと区別するためにユーザの個人情報にアクセスする箇所などを赤く表示する.
マルウェアが外部からコードをダウンロードして攻撃をする場合,コード解析では対応できないという問題点がある.
droidbox はエミュレータ上でマルウェアを動作させることで,データの送受信,ファイルの読み書きなどを動的に監視することでマルウェアの挙動を解析している.
しかし,droidbox はエミュレータ上でマルウェアを動作させているために,実機上でのマルウェアの挙動を正確に解析できない場合がある.
なぜならマルウェアがエミュレータ上で動作していることを検知して,マルウェア自身が挙動を変える可能性もあるためである.

そこで本研究では実機における Android マルウェアの動的解析手法を提案する.
マルウェアの実際の挙動をより詳細に調べるためには,実機上でマルウェアを動作させて,その挙動から解析を行う必要がある.
マルウェアの挙動を解析するために,実機上でログを出力するコードをマルウェアに実行させてログを得る.
%need to consider this above sentence
提案手法をマルウェアに適用することで,実行されたメソッド名,クラス名,引数の型名と値をログとして獲得することができる.

マルウェアの実行時にログを得るために,マルウェアのソースコードのソースコードを書き換える.
マルウェアである Android アプリケーションは APK ファイルという1つのファイルにまとめられて端末にインストールされている.
その APK ファイルから Java クラスファイルを取り出して,ログを得たいメソッドを含むクラスの Java クラスファイルを書き換える.
Java クラスファイルを書き換えたマルウェアの APK ファイルを実機にインストールして動かすと,動的にログを得ることができる.
このログを得ることでマルウェアのどのクラス,メソッドが不正な動きをしているのか,マルウェアがどのような情報にアクセスしているかがわかるため,解析を行うことができる.

提案手法によりマルウェアを解析できることを示すためにインターネットのサイト\cite{malwaresite}から入手した 11 個 のマルウェアを用いて 2 種類の実験を行った.
1 つめの実験では,11 個の検体において,不正なコードを含むと思われるクラスのそれぞれのメソッドの始めにログを出力するようにクラスファイルを変更した.
その結果,11 個中 5 個のマルウェアから,不正な挙動を表すログを得ることができた,
具体的にいうと,SMS の送信や外部からのコードの入手を示すログであった.
2 つめの実験では,先の 11 個の検体の中の 1 つである iMatch に対して行った,
1 つめの実験で行ったクラスファイルの変更に加えて,メソッド内でのメソッド呼び出しの情報も出力するようにした.
この実験の結果として,iMatch の攻撃手段である,SMS 送信のための Android API とそのメソッドを呼び出しているメソッドとそのクラスを特定することができた.
2 つの実験を通じて提案手法により 5 個ののマルウェアの挙動を解析することができた.

本論文の構成を以下に示す.
2 章では Android 端末を標的にした悪意あるマルウェアと Android アプリケーションの基本的な構成について解説する.
3 章では,Android アプリケーションのマルウェアを解析している関連研究を紹介する.
4 章では,マルウェアを解析するためにどのようにマルウェアの中にログコードを挿入するかについて説明する.
\ref{sec:exp} 章では,提案手法をマルウェアに適用させた実験とその結果について述べる.
\ref{sec:concl} 章では,まとめと今後の課題について考察する.