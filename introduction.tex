\section{はじめに}
\label{sec:intro}
今日では,スマートフォンが非常に身近な存在であり,その中でも Android 端末は最も世界中で普及している. IDC による,世界中のスマートフォンの OS 別のシェアの調査\cite{osshare} においては,Android 端末は80\% 以上のシェアがあると示している. つまり,Android 端末は他の OS の端末,iOS, Windows Phone, Black Berry OS に比べて,より多くのユーザによって使われていることがわかる.Android がオープンソースであり,だれでも Android のソースコードを手に入れられることがその理由の1つである.そのため,様々なメーカーによって開発が行われ,価格,スペックに関して多種多様な製品が世界中で販売されている.

しかし,Androidの普及に伴い,Android 端末を標的にしたAndroid アプリのマルウェアによる被害が増えている.先に述べたように,Android はオープンソースであるため,攻撃者は脆弱性を見つけることは 他のモバイル端末 OS に比べると容易だ.Cisco の 2014 年のレポート\cite{cisco} によると,モバイル向けマルウェアの 99.9\% が Android を標的にしていると報告している.Y.Zhou, X.Jiang の研究\cite{dissect} では,彼らの研究に用いたデータセットの数は,2011 年の6月では 209 個だったのが,同年 10 月には 1260 個にも増加していた.2014 年の 9 月には,ロシアで銀行口座を狙った Android マルウェアを作成したとして,2名の逮捕者が出ている.これらの事例を見てわかるように,Android 端末を標的にしたマルウェアによる被害は深刻であり,Android 端末のユーザは危険にさられている

これまで解析された多くの Android マルウェアが次の 2 つの挙動を示している.1 つは,個人,または端末の情報を抜き取ることである.感染した端末の情報を内部で収集し,この情報を外部へ送信する.例えば感染した端末に保存されている電話帳データを盗むことによって,マルウェア作成者は他の端末への攻撃が可能になる.もうひとつは,ユーザへの不正な課金である.バックグラウンドで SMS を送ることで,ユーザは不正に金額を請求させられてしまう.さらにマルウェアはユーザにメッセージを送信したことを気づかせないようにするので,ユーザは請求が来るまで全く気づかない.もちろん Android マルウェアはこれ以外にもさまざまな挙動を示す.このような被害を少なくするためには Android を標的にしたマルウェアを解析することはとても重要である.

code.google.com が提供している既存のAndroid マルウェアの解析ツールとして androguard\cite{aguard} , droidbox\cite{dbox} の2つがある.androguard は Android アプリのコード解析を行うことで,アプリ内のクラスごとの関係を示すグラフを作り,危険だと判定した部分を赤く表示する.マルウェアが外部からコードをダウンロードして攻撃をする場合,静的解析では対応できないという問題点がある.droidbox はエミュレータ上でマルウェアを動かし,データのやりとり,ファイルの読み書き,などを動的に監視することでマルウェアの挙動を解析している.しかし,droidbox はマルウェアが実際に実機でどのような挙動をするかを正確にとらえているとは限らない.なぜならマルウェアがエミュレータ上で動いているのを検知して,挙動を変える可能性もあるからだ.

そこで本研究では実機における Android マルウェアの動的解析を提案する.マルウェアの実際の挙動をより詳細に調べるためには,実機でマルウェアを動かし,その挙動から解析を行う必要がある.マルウェアを実機で動かしながら,ログを得ることで解析を行う.提案手法をマルウェアに適用することで,実行されたメソッド名,クラス名,引数の型名と値を得ることができる.Android アプリは APK ファイルという1つのファイルにまとめられて端末にインストールされている.その APK ファイルから Java クラスファイルを取り出して,ログを得たいメソッドを含むクラスの Java クラスファイルを書き換える.Java クラスファイルを書き換えたマルウェアの APK ファイルを実機にインストールして動かすと,動的にログを得ることができる.このログを得ることでマルウェアのどのクラス,メソッドが不正な動きをしているのか,マルウェアがどのような情報にアクセスしているかがわかるため,解析を行うことができる.

本提案によりマルウェアを解析できたかを示すためにインターネットのサイトから入手した 11 個 のマルウェアを用いて 2 種類の実験を行った.1 つめの実験では,11 個の検体において,不正なコードを含むと思われるクラスのそれぞれのメソッドの始めにログを出力するようにクラスファイルを変更した.その結果,11 個中 5 個のマルウェアから,不正な挙動を表すログを得ることができた,具体的にいうと,SMS の送信や,外部からのコードの入手を示すログであった.2 つめの実験では,先の 11 個の検体の中の 1 つである,iMatch に対してのみ行った,1 つめの実験で行ったクラスファイルの変更に加えて,メソッド内でのメソッド呼び出しの情報も出力するようにした.この実験の結果として,iMatch の攻撃手段である,SMS 送信のための Android API とそのメソッドを呼び出しているメソッドとそのクラスを特定することができた.2 つの実験を通じて提案手法により一部のマルウェアの挙動を解析することができた.

本論文の構成を以下に示す.2 章では Android 端末を標的にした悪意あるマルウェアと Android アプリの基本的な構成について解説する.3 章では,Android アプリのマルウェアを解析している関連研究を紹介する.4 章では,マルウェアを解析するためにどのようにマルウェアの中にログコードを挿入するかについて説明する.\ref{sec:exp} 章では,提案手法をマルウェアに適用させた実験とその結果について述べる.\ref{sec:concl} 章では,まとめと今後の課題について考察する.