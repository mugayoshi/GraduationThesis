\section{実装}
\label{sec:instrument}
本章では,本提案をマルウェアに適用するために実装したプログラムについて詳しく説明する.
\ref{programforclass}では,Java ウラスファイルを書き換えるためのプログラムについて,\ref{splitscript}では,DDMS で得られたログを処理するためのプログラムについて擬似コードやソースコードを交えながらそのアルゴリズムや実行の流れを解説する.

\subsection{Java クラスファイルを書き換えるためのプログラム}
\label{programforclass}
\ref{method top}, \ref{methodicalls} で提案した手法を実現するためには,Javaクラスファイルを操作したり,中身を見る必要がある,
そのためには,Javassist\cite{javassist} を用いたプログラム (Java で実装したプログラム) が必要である.
Java クラスファイルはバイナリーコードではないため,読むことはできる.
しかし,Java ファイル (.java) の中身と対応させるためには,Java バイトコードの知識が必要である.
さらに,jad\cite{jad}というツールを使うことで,Java クラスファイルを Java ファイルに逆コンパイル (.class → .java) することもできる.



\ref{methodtop}と,\ref{methodcalls} で提案する手法では,それぞれ Java クラスファイルを操作するアプローチが異なるため,別々に説明する.

\subsubsection{メソッドへのコードの挿入}
マルウェアに限らず,Android のアプリのソースコードは多くのクラスから構成されている.例えば,ある辞書アプリのクラスファイルの数は 120 を超える.このような多数のクラスファイルのメソッドをひとつひとつ処理するのは効率が悪い.また,
いくつものクラスファイルのメソッドへコードを挿入するためのプログラムについて説明する.

全体の流れを説明する.クラス名を入力する.
クラスを検索する.
そのリストにあるクラスファイルを書き換える.
	

流れ

\subsubsection{メソッド呼び出しの置き換え}


\subsubsection{private メソッドを書き換える方法}
\label{private}

\subsection{得られたログを処理するためのプログラム}
\label{splitscript}