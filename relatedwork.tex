\section{関連研究}
 Android マルウェアを解析,調査した研究を 3 つ紹介する.

Y.Zhou, X.Jiang は,2010 年の 8 月から 2011 年の10 月にかけて,公式サイト,非公式サイトから収集した 1,260 個,49 種類のAndroid マルウェアを用いて時系列調査と分類調査を行っている \cite{dissect} .時系列調査では,この研究で収集しているマルウェアの数 (dataset) をグラフで示している.DroidKungFu が登場した 2011 年 6 月,AnserverBot が登場した 2011 年 10 月にこの dataset の数が急激に増えた.つまり,この 2 つのマルウェアが大きな影響を及ぼしていることがわかる.分類調査では,マルウェアのインストールの方法,起動トリガー,挙動,マルウェアが要求する権限を調査している.49 種類中,25 種類のマルウェアが Repackage によりインストールされていた.マルウェアの起動トリガーとして最も多かったのは,OS の起動時で,29 種類だった.マルウェアの挙動は,Financial charge が最も多く,その中でも,SMS を使っているものが多かった.また,多くのマルウェアが SMS, Wi-Fi に関する権限を要求していた.また,先に出てきた,2 つのマルウェアがどのような挙動をするかを示している.DroidKungFu は6 種類のヴァージョンが見つかっており,外部サーバのアドレスの格納方法がジョジョジョに複雑になっている.AnserverBot は解析回避と遠隔操作の2 つの特徴を持つ.さらに,既存の 4つのセキュリティソフトがこの  dataset を検知するかどうかの調査も行った.その結果,最高は Lookout の 79.6 \% , 最低は Norton の 20.2 \% で,どのソフトウェアも検知できないマルウェアも存在した.この結果より,この 4 つのセキュリティソフトはまだ不十分であることがわかる.これらの調査結果からこの研究の結論として,マルウェアのインストール方法の中でも,最も頻繁に行われている Repackage を検知することとアプリの外部のコードの動的ローディングを防ぐ技術が必要であると彼らは主張している.この研究は Android マルウェアを調査,分類し,さらに 2 つのマルウェアについて詳しく挙動を示している.この研究では,解析手段(静的か動的か)までは言及していない.本研究では調査,分類は行わず,解析のみを行っている.

S.Poeplau らは,マルウェアの動的に外部コードの動的ローディングに焦点をおいて解析を行っている \cite{dynamicloading} . \ref{sec:malware} で述べたように,外部コードのローディングをすることで公式ストアの検知システムをくぐり抜けることができる.また,外部コードのローディングは必ずしも不正なものではなく,マルウェア以外のアプリでも使われている.しかし,Android OS はロードされたコードをチェックしないので攻撃者はロードするものを置き換えることができる.そのためこれは Android アプリの脆弱性といえる.そこで,この研究で彼らは外部コードの動的ローディングを検知するツールを提案している.このツールは APK ファイルから取り出した DEX コードを静的に解析する.100 万回以上インストールされた,1,632 個のアプリをランダムに選び,このツールを用いて検査した.その結果,その中の 9.25 \% から外部コードのローディングの脆弱性が検知された.さらに,Google Play  での人気 50 位以内の無料アプリを同じツールで検査すると,16 \% ものアプリがその脆弱性を示した.また,彼らはこの攻撃手法に対する防御策も提案している.Dalvik VM が外部からダウンロードされたコードのハッシュ値を計算し,それが Whitelist に載っていなければ,それを実行できないようにアプリケーションに制限をかける.そうすることで,これを利用した攻撃を防ぐことができる.この研究での外部コードのローディングを検知するツールは,静的に行っているため,マルウェアを全く動かしていない.それに対して本研究ではマルウェアを実機で実際に動かして動的にログを得ることで解析している.また,この研究では,ひとつの攻撃手法に限定して解析を行っているが,本研究では,特定の攻撃手法に限定していない.

L.Yan, H.Yin はマルウェアの動的解析環境 (DroidScope) を提案している \cite{droidscope} .彼らは Android SDK が提供するエミュレータをベースに自分たちで手を加えて,そのエミュレータの中でマルウェアを動かしている.彼らは Android システム全体の再構築を行っている.つまり,DroidScope はハードウェア,Linux OS (Android は Linux をベースにして動作している),Dalvik VM の 3 種類の API を提供する.DroidScope が提供する API を用いることで,Android API, Dalvik VM, Linux , さらには機械語の命令までをトレースするツールを提案している.{\it API tracer, native tracer, Dalvik instruction tracer} の 3 つである.また,これらの API に動的な taint analysis を実行することで,情報漏えいを解析するツール ({\it Taint tracker}) も提案している.彼らはこの 4 つのツールのパフォーマンス測定も行っている.元のエミュレータで実行時間を基準に,4 つのツールの実行時間を調べた.その結果,オーバーヘッドは小さいと言える結果であった.しかし,taint traker は他の 3 つのツールとくらべて大きなオーバーヘッドを示した.彼らはこれらのツールを用いて先に述べた,DroidKungFu を解析した.この解析によって,このマルウェアのルート権限を取得する方法と情報を盗み出す方法を明らかにしている.DroidKungFu だけでなく,論文中では DroidDream も解析している.DroidKungFu の場合と同様な解析を行った結果,DroidDream が端末識別番号を盗み出す方法を突き止めた.彼らの研究は実行された API,メソッドを動的に得ることで解析を行っているため,解析のためのアプローチは本研究と共通している点がある.しかし,本研究は実機で行っているのに対して,彼らの研究はエミュレータ上で行っている.もし,マルウェアがエミュレータ上で動作しているのを検知して,振る舞いを変える可能性もある.そのため,実機で解析すればマルウェアの "正常" な動作を解析することができる.

ここで紹介した研究には問題点が残っている.外部コードの動的ローディングの応用例として,文字列としてクラス名,メソッド名を受け取り,  reflection を通して実行することが考えられる.S.Poelau らの研究では外部コードの動的ローディングの攻撃を防ぐツールを提案したが,reflection を使うと,このツールでは検知することができない.DroidScope は動的解析のため,コードカバレッジが制限される.実行時には,1 つの実行パスしか通ることはない.L.Yan, H.Yin は実行パスを増やすために,システムコール,ネイティブ API,Dalvik メソッドなどの返り値を変えることで,異なる実行パスを実現した.symbolic execution のほうが,より良いことが考えられるが,かれらはこれを今後の課題としている.


