\section{おわりに}
\label{sec:concl}
\subsection{まとめ}
本論文では,Android 端末を標的にしたマルウェアを動的に解析した.
マルウェアの Java クラスファイルにログコードを挿入し,ログコードを挿入したマルウェアを実機の Android 端末で実行することで動的にログを得た.

%本提案を実際に集取したマルウェアに対して適用した.
本提案では,2 つの手法によりログコードを挿入する.
メソッドの先頭にコードを挿入する手法とメソッド呼び出しの前後に挿入する手法である.
11 個のマルウェアに対して 1 つ目の手法を適用し,その中の 1 つのマルウェア, iMatch のみに 2 つめの手法を適用した.
この 2 つの手法によりマルウェアにログを挿入して実機で実行した.

11 個のマルウェアにコードを挿入して実行したところ,その中の 5 個から不正な動きを示すログを得ることができた.
その他のマルウェアからは特定の挙動を示すログを得ることはできなかった.
ログを得た 5 つマルウェアの 2 つは SMS を送信しようとしていた.
もう一つのマルウェアからは外部からコードを受け取るメソッドのログを得ることができた.

さらに,iMatch が SMS を送信する過程を明らかにすることができた.
SMS を送信するメソッドがどこから実行されるか,どのタイミングで実行されるかを特定できた.
このメソッドはこのマルウェアを起動したときに実行され.Android API を使って,SMS を送信していることがわかった.

いくつかのマルウェアからログを得ることができなかった原因として考えられるのは,SIM カードと外部サーバの問題である.
実験を行った端末は SIM カードが挿入していなかったため,電話の発着信と SMS の送受信の機能を持っていなかった.
そのため,これらのシステムイベントを発生させることができず,このイベントを監視しているマルウェアが起動しなかった可能性がある.
本研究で扱ったマルウェアは数年前に作成されたものであり,決して新しいものとはいえない.
よって,マルウェアが通信を行うサーバが動作していなかったために,外部サーバとのやりとりを行うログを得ることができなかったと考えられる.

\subsection{今後の課題}
\subsubsection{他のマルウェアへの本提案手法の適用}
今後の課題のひとつとして,他のマルウェアにも\ref{methodcalls} の手法を適用することが挙げられる.
本研究では,\ref{methodcalls}を適用したのは iMatch のみであった.
実験結果でも示したように,この手法では\ref{methodtop}の手法と比べて,マルウェアのより詳細な挙動を解析することができる.
そのため,本研究で取り扱った他のマルウェアに対してこの手法を適用することは有効であるといえる.

\subsubsection{SIM カード}
SIM カードを挿入することで,本研究でログを得られなかったマルウェアからログを得られる.
\ref{leg}のログを出力した Beauty Leg をはじめ,電話の着信や SMS の受信を監視しているマルウェアがいくつもある.
\ref{expmalware}で挙げたように,Beauty Leg, Beauty Breast, Beauty Girl は電話の着信により起動する.
SIM カードを挿入した状態でこの端末に電話を発信すれば,これらのマルウェアが起動し,ログを得られる.
また,SIM カードを挿入すれば,SMS を送受信することができる.
SIM カードを挿入して iMatch を実行すると,本研究の実験結果とは異なる挙動を見せる可能性がある.

\subsubsection{Android API のログ}
マルウェアが呼び出す Android API についての情報を得ることができれば,さらに詳細な挙動を解析できる.
本研究では,iMatch がどのように SMS を送るかを明らかにできた.
しかし,Android API の引数の情報を得ることはできなかったので,SMS の送り先や中身まではわからなかった.
そのためには Android API のソースコードにログコードを書き加え,書き加えた API を Android OS で動作させればよい.
つまり,オリジナルの Android OS をビルドするということである.
その OS の中でマルウェアを動かせば Android API のログを得ることができる.