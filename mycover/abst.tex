\setlength{\parindent}{10pt}

現在,Android 端末は世界中で広く用いられている.
しかし,Android 端末を標的とした悪意ある Android アプリのマルウェアによる被害が多数発生している.
Android アプリのマルウェアは感染した端末の個人情報を盗み取って外部サーバへ送信したり,バックグラウンドでメッセージを送信することでユーザに気づかれることなく不正な課金を行う.
このような Android アプリのマルウェアによる被害を防ぐためにはマルウェアの解析が必要である.


Android アプリのマルウェアを解析する方法の 1 つとして,マルウェアを実際に動作させて解析を行う動的解析がある.
%静的解析は,,,?
既存の動的解析手法・ツールではエミュレータ環境において解析を行っている.
そのため,本研究では実機の Android 端末を用いて動的解析を行う.

本研究ではマルウェアの Java クラスファイルにログコードを挿入し,その実行ログを得ることで,動的に解析を行う.
%マルウェの Java クラスファイルは既存のツールを用いたリバースエンジニアリングにより入手する.
Java クラスファイルを書き換える Java ライブラリを用いることで,マルウェアのクラスファイルにログコードを挿入する.

提案手法によってマルウェアを解析できるか実験を行う.
提案手法を 11 個のマルウェアに適用し,Android の実機 (Nexus 5 Android 4.4.4) でこのマルウェアを実行した.
その中の 1 つのマルウェアから外部からコードを入手するメソッドを,2 つのマルウェアからは SMS を送信するメソッドのログを得ることができた.
さらに,iMatch というマルウェアが SMS を不正に送信する方法を特定することができた.